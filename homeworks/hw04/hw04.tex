\documentclass[12pt]{article}

\include{preamble}

\newtoggle{professormode}
\toggletrue{professormode} %STUDENTS: DELETE or COMMENT this line



\title{MATH 621 Fall 2017 Homework \#4 INCOMPLETE}

\author{Professor Adam Kapelner} %STUDENTS: write your name here

\iftoggle{professormode}{
\date{Due \textit{in review session} or KY604 9:30PM, Monday, November 13, 2017 \\ \vspace{0.5cm} \small (this document last updated \today ~at \currenttime)}
}

\renewcommand{\abstractname}{Instructions and Philosophy}

\begin{document}
\maketitle

\iftoggle{professormode}{
\begin{abstract}
The path to success in this class is to do many problems. Unlike other courses, exclusively doing reading(s) will not help. Coming to lecture is akin to watching workout videos; thinking about and solving problems on your own is the actual ``working out.''  Feel free to \qu{work out} with others; \textbf{I want you to work on this in groups.}

Reading is still \textit{required}. For this homework set, read about transformations of r.v.'s (discrete and continuous).

The problems below are color coded: \ingreen{green} problems are considered \textit{easy} and marked \qu{[easy]}; \inorange{yellow} problems are considered \textit{intermediate} and marked \qu{[harder]}, \inred{red} problems are considered \textit{difficult} and marked \qu{[difficult]} and \inpurple{purple} problems are extra credit. The \textit{easy} problems are intended to be ``giveaways'' if you went to class. Do as much as you can of the others; I expect you to at least attempt the \textit{difficult} problems. 

This homework is worth 100 points but the point distribution will not be determined until after the due date. See syllabus for the policy on late homework.

Up to 10 points are given as a bonus if the homework is typed using \LaTeX. Links to instaling \LaTeX~and program for compiling \LaTeX~is found on the syllabus. You are encouraged to use \url{overleaf.com}. If you are handing in homework this way, read the comments in the code; there are two lines to comment out and you should replace my name with yours and write your section. The easiest way to use overleaf is to copy the raw text from hwxx.tex and preamble.tex into two new overleaf tex files with the same name. If you are asked to make drawings, you can take a picture of your handwritten drawing and insert them as figures or leave space using the \qu{$\backslash$vspace} command and draw them in after printing or attach them stapled.

The document is available with spaces for you to write your answers. If not using \LaTeX, print this document and write in your answers. I do not accept homeworks which are \textit{not} on this printout. Keep this first page printed for your records.

\end{abstract}

\thispagestyle{empty}
\vspace{1cm}
NAME: \line(1,0){380}
\clearpage
}


\problem{We will practice transformations on discrete r.v.'s.}



\begin{enumerate}

\easysubproblem{If $X \sim \binomial{n}{p}$ and $Y = c + aX$, find $p_Y(y)$ and the $\support{Y}$.}\spc{2}


\easysubproblem{If $X \sim \binomial{n}{p}$ and $Y = X^2$, find $p_Y(y)$ and the $\support{Y}$.}\spc{3}


\hardsubproblem{If $X \sim \binomial{n}{p}$ and $Y = |X - k|$ where $k \in \naturals$, find $p_Y(y)$ and the $\support{Y}$.}\spc{12}


\hardsubproblem{If $X \sim \poisson{\lambda_1}$ independent of $Y \sim \poisson{\lambda_2}$, and $T = X - Y$, find $p_T(t)$ and the $\support{T}$. This is called the Skellam$(\lambda_1, \lambda_2)$ distribution, the generalization of what we found in class.}\spc{13}

\intermediatesubproblem{According to the \href{https://www.baseball-reference.com/leagues/MLB/2016.shtml}{major league baseball statistics of 2016}, the New York Yankees scored 4.20 runs per game on average and the Boston Red Socks scored 5.42 runs per game on average. If the Poisson distribution is a good model for runs scored in a baseball game (and that's a big \qu{if}), find an expression for the probability the Yankees beat the Red Sox in one game.}\spc{10}

\easysubproblem{Evaluate the expression. You will need the \texttt{besselI} \href{https://stat.ethz.ch/R-manual/R-devel/library/base/html/Bessel.html}{function in R}.}\spc{3}
 
\end{enumerate}

\problem{We will now practice transformations on continuous r.v.'s.}

\begin{enumerate}

\easysubproblem{If $X \sim \exponential{\lambda}$ and $Y = c + aX$, find $f_Y(y)$ and the $\support{Y}$.}\spc{2}

\intermediatesubproblem{If $X \sim \erlang{k}{\lambda}$ and $Y = aX$ where $a \in (0, \infty)$, find $f_Y(y)$ and the $\support{Y}$. If possible, give its brand name and find its parameters.}\spc{17}

\intermediatesubproblem{If $X \sim \exponential{\lambda}$ and $Y = e^{-X}$, find $f_Y(y)$ and the $\support{Y}$.}\spc{10}

\intermediatesubproblem{If $X \sim \exponential{1}$ and $Y =-\natlog{\frac{e^{-X}}{1 - e^{-X}} }$, find $f_Y(y)$ and the $\support{Y}$.}\spc{10}

\end{enumerate}

\problem{The following exercises are on survival distributions.}

\begin{enumerate}

\easysubproblem{If $X \sim \exponential{\lambda}$ and $Y =-\natlog{X}$, find $f_Y(y)$ and the $\support{Y}$ and if it's a brand name r.v. indicate so.}\spc{9}

\easysubproblem{If $X \sim \gumbel{\mu}{1}$ and $Y =e^{-X}$, find $f_Y(y)$ and the $\support{Y}$ and if it's a brand name r.v. indicate so.}\spc{9}

\hardsubproblem{If $X \sim \exponential{1}$ and $Y =\mu + \natlog{\oneover{X^\beta}}$, find $f_Y(y)$ and the $\support{Y}$ and if it's a brand name r.v. indicate so.}\spc{9}

\intermediatesubproblem{Let $X \sim \weibull{k}{\lambda}$. In class we showed that if $k=2$, the r.v. is not memoryless; the probability decreases as you condition on longer and longer survival times i.e. $\prob{X \geq b} > \cprob{X \geq a + b}{X \geq a}$. To prove this for arbitrary $k>1$, first show that this implies $a^k + b^k < \tothepow{a+b}{k}$.}\spc{9}

\easysubproblem{Let $X \sim \weibull{k}{\lambda}$. In class we showed that if $k=\half$, the r.v. is not memoryless; the probability increases as you condition on longer and longer survival times i.e. $\prob{X \geq b} < \cprob{X \geq a + b}{X \geq a}$. To prove this for arbitrary $k<1$, first show that this implies $a^k + b^k > \tothepow{a+b}{k}$. Hint: this is the same problem as (d) except the inequality flips.}\spc{9}

\extracreditsubproblem{Prove (d) and (e).}\spc{9}


\hardsubproblem{If $X \sim \stduniform$ and $Y = \beta\tothepow{-\natlog{X}}{-\oneover{\alpha}}$, find $f_Y(y)$ and the $\support{Y}$ and if it's a brand name r.v. indicate so.}\spc{14}


\easysubproblem{If $X \sim \erlang{k}{\lambda}$ explain why allowing $k \in (0, \infty)$ instead of $k \in \naturals$ is reasonable. What is this \qu{new} r.v. called? (It is not a transformation of the Erlang, it is just expanding its parameter space and hence generalizing it a little bit).}\spc{9}


\end{enumerate}

\end{document}
