\documentclass[12pt]{article}

\include{preamble}

\newtoggle{professormode}
\toggletrue{professormode} %STUDENTS: DELETE or COMMENT this line



\title{MATH 621 Fall 2017 Homework \#5 INCOMPLETE}

\author{Professor Adam Kapelner} %STUDENTS: write your name here

\iftoggle{professormode}{
\date{Due KY258 8PM, Thursday, November 30, 2017 \\ \vspace{0.5cm} \small (this document last updated \today ~at \currenttime)}
}

\renewcommand{\abstractname}{Instructions and Philosophy}

\begin{document}
\maketitle

\iftoggle{professormode}{
\begin{abstract}
The path to success in this class is to do many problems. Unlike other courses, exclusively doing reading(s) will not help. Coming to lecture is akin to watching workout videos; thinking about and solving problems on your own is the actual ``working out.''  Feel free to \qu{work out} with others; \textbf{I want you to work on this in groups.}

Reading is still \textit{required}. For this homework set, read about the normal distribution, the $\chi^2$ distribution, Student's $T$, the $F$, Cochran's Theorem, the multivariate normal distribution and characteristic functions.

The problems below are color coded: \ingreen{green} problems are considered \textit{easy} and marked \qu{[easy]}; \inorange{yellow} problems are considered \textit{intermediate} and marked \qu{[harder]}, \inred{red} problems are considered \textit{difficult} and marked \qu{[difficult]} and \inpurple{purple} problems are extra credit. The \textit{easy} problems are intended to be ``giveaways'' if you went to class. Do as much as you can of the others; I expect you to at least attempt the \textit{difficult} problems. 

This homework is worth 100 points but the point distribution will not be determined until after the due date. See syllabus for the policy on late homework.

Up to 10 points are given as a bonus if the homework is typed using \LaTeX. Links to instaling \LaTeX~and program for compiling \LaTeX~is found on the syllabus. You are encouraged to use \url{overleaf.com}. If you are handing in homework this way, read the comments in the code; there are two lines to comment out and you should replace my name with yours and write your section. The easiest way to use overleaf is to copy the raw text from hwxx.tex and preamble.tex into two new overleaf tex files with the same name. If you are asked to make drawings, you can take a picture of your handwritten drawing and insert them as figures or leave space using the \qu{$\backslash$vspace} command and draw them in after printing or attach them stapled.

The document is available with spaces for you to write your answers. If not using \LaTeX, print this document and write in your answers. I do not accept homeworks which are \textit{not} on this printout. Keep this first page printed for your records.

\end{abstract}

\thispagestyle{empty}
\vspace{1cm}
NAME: \line(1,0){380}
\clearpage
}






\problem{The $\chi^2$ r.v. within Cochran's Theorem.}

\begin{enumerate}


\easysubproblem{Given $\Xoneton \iid f(\mu,\sigsq)$, a density with finite variance, state the classic estimator (a r.v.) and the estimate (a scalar value) for $\mu$.}\spc{1}


\easysubproblem{Prove this estimator is unbiased i.e $\expe{\cdot} = \mu$.}\spc{2}

\easysubproblem{Given $\Xoneton \iid f(\mu,\sigsq)$, a density with finite variance, state the classic estimator $S^2$ (a r.v.) and the estimate (a scalar value) for $\sigsq$.}\spc{1}

\hardsubproblem{Prove this estimator is unbiased i.e $\expe{\cdot} = \sigsq$. The answer is online but try to do it yourself.}\spc{12}


\easysubproblem{State Cochran's Theorem.}\spc{6}

\easysubproblem{Given $\Xoneton \iid \normnot{\mu}{\sigsq}$ Show that $\sum_{i=1}^n \squared{\frac{X_i - \mu}{\sigma}} \sim \chisq{n}$.}\spc{1}

\easysubproblem{Express $\sum_{i=1}^n \squared{\frac{X_i - \mu}{\sigma}}$ in vector notation.}\spc{1}

\easysubproblem{Express $\sum_{i=1}^n \squared{\frac{X_i - \mu}{\sigma}}$ as a quadratic form. What is the matrix that determines this quadratic form?}\spc{1}

\easysubproblem{What is the rank of the determining matrix?}\spc{1}

\easysubproblem{When computing $\sum_{i=1}^n \squared{\frac{X_i - \mu}{\sigma}}$, how many \href{https://en.wikipedia.org/wiki/Degrees_of_freedom_(statistics)}{independent pieces of information} go into the calculation?}\spc{1}

\easysubproblem{Define degrees of freedom.}\spc{1}

\easysubproblem{Show that $\sum_{i=1}^n \squared{\frac{X_i - \mu}{\sigma}} = \frac{(n-1)S^2}{\sigsq} + \frac{n(\Xbar - \mu)}{\sigsq}$.}\spc{4}


\easysubproblem{Show that $\frac{n(\Xbar - \mu)}{\sigsq} \sim \chisq{1}$.}\spc{4}

\easysubproblem{Express $\frac{n(\Xbar - \mu)}{\sigsq}$ in vector notation.}\spc{4}

\easysubproblem{Express $\frac{n(\Xbar - \mu)}{\sigsq}$ as a quadratic form. What is the matrix that determines this quadratic form? Call it $B_2$.}\spc{4}

\easysubproblem{What is the rank of the determining matrix?}\spc{1}

\easysubproblem{When computing $\frac{n(\Xbar - \mu)}{\sigsq}$, how many \href{https://en.wikipedia.org/wiki/Degrees_of_freedom_(statistics)}{independent pieces of information} go into the calculation?}\spc{1}

\easysubproblem{Express $\frac{(n-1)S^2}{\sigsq}$ in vector notation.}\spc{4}

\intermediatesubproblem{Express $\frac{(n-1)S^2}{\sigsq}$ as a quadratic form. What is the matrix that determines this quadratic form? Call it $B_1$.}\spc{4}

\intermediatesubproblem{What is the rank of the determining matrix?}\spc{1}

\easysubproblem{When computing $\frac{(n-1)S^2}{\sigsq}$, how many \href{https://en.wikipedia.org/wiki/Degrees_of_freedom_(statistics)}{independent pieces of information} go into the calculation?}\spc{1}

\easysubproblem{What is $B_1 + B_2$? Why should this be?}\spc{3}


\easysubproblem{What is rank$(B_1)~+~$rank$(B_2)$?}\spc{3}

\easysubproblem{Show that $B_1$ is positive semi-definite (PSD).}\spc{3}

\easysubproblem{Show that $B_2$ is positive semi-definite (PSD).}\spc{3}

\intermediatesubproblem{Using Cochran's Theorem, show that $\frac{(n-1)S^2}{\sigsq} \sim \chisq{n-1}$ and that $\frac{(n-1)S^2}{\sigsq}$ is independent of $\frac{n(\Xbar - \mu)}{\sigsq}$.}\spc{6}


\hardsubproblem{What is $B_1B_2$? Why should this be?}\spc{3}

\intermediatesubproblem{Using the answer in (z), show that $\frac{\Xbar - \mu}{\oversqrtn{S}} \sim T_{n-1}$.}\spc{6}

\hardsubproblem{In (d), you proved that $\expe{S^2} = \sigsq$. What is $\expe{S}$ under the assumption of $\Xoneton \iid \normnot{\mu}{\sigsq}$? You will need to read online about the $\chi$ distribution.}\spc{5}

\hardsubproblem{Create a new estimator $S'$ that is unbiased for estimating $\sigma$. You can use a function of the original $S$.}\spc{8}

\end{enumerate}

%\problem{Some questions about ch.f.'s.}
%
%\begin{enumerate}
%
%\easysubproblem{}\spc{2}
%\end{enumerate}

\end{document}