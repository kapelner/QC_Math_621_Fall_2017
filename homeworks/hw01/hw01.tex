\documentclass[12pt]{article}

\include{preamble}

\newtoggle{professormode}
\toggletrue{professormode} %STUDENTS: DELETE or COMMENT this line



\title{MATH 621 Fall 2017 Homework \#1}

\author{Professor Adam Kapelner} %STUDENTS: write your name here

\iftoggle{professormode}{
\date{Due \emph{in class}, Thursday, September 14, 2017 \\ \vspace{0.5cm} \small (this document last updated \today ~at \currenttime)}
}

\renewcommand{\abstractname}{Instructions and Philosophy}

\begin{document}
\maketitle

\iftoggle{professormode}{
\begin{abstract}
The path to success in this class is to do many problems. Unlike other courses, exclusively doing reading(s) will not help. Coming to lecture is akin to watching workout videos; thinking about and solving problems on your own is the actual ``working out.''  Feel free to \qu{work out} with others; \textbf{I want you to work on this in groups.}

Reading is still \textit{required}. For this homework set, review Math 241 concerning random variables, support, parameter space, PMF's, CDF's. Then read about convolutions and the multinomial distribution.

The problems below are color coded: \ingreen{green} problems are considered \textit{easy} and marked \qu{[easy]}; \inorange{yellow} problems are considered \textit{intermediate} and marked \qu{[harder]}, \inred{red} problems are considered \textit{difficult} and marked \qu{[difficult]} and \inpurple{purple} problems are extra credit. The \textit{easy} problems are intended to be ``giveaways'' if you went to class. Do as much as you can of the others; I expect you to at least attempt the \textit{difficult} problems. 

This homework is worth 100 points but the point distribution will not be determined until after the due date. See syllabus for the policy on late homework.

Up to 10 points are given as a bonus if the homework is typed using \LaTeX. Links to instaling \LaTeX~and program for compiling \LaTeX~is found on the syllabus. You are encouraged to use \url{overleaf.com}. If you are handing in homework this way, read the comments in the code; there are two lines to comment out and you should replace my name with yours and write your section. The easiest way to use overleaf is to copy the raw text from hwxx.tex and preamble.tex into two new overleaf tex files with the same name. If you are asked to make drawings, you can take a picture of your handwritten drawing and insert them as figures or leave space using the \qu{$\backslash$vspace} command and draw them in after printing or attach them stapled.

The document is available with spaces for you to write your answers. If not using \LaTeX, print this document and write in your answers. I do not accept homeworks which are \textit{not} on this printout. Keep this first page printed for your records.

\end{abstract}

\thispagestyle{empty}
\vspace{1cm}
NAME: \line(1,0){380}
\clearpage
}


\problem{These exercises review convolutions.}


\begin{enumerate}

\intermediatesubproblem{Let $X_1, X_2 \iid \bernoulli{p}$. Using the definition of convolution without the indicator function within $p_{X_2}$, write out all terms of the sum and identify which one is illegal.}\spc{4}

\easysubproblem{Let $X_1 \sim \bernoulli{p_1}$ independent of $X_2 \sim \bernoulli{p_2}$. Find the PMF of the sum of $T = X_1 + X_2$ using basic probability theory. You can denote the PMF using a piecewise function (with cases).}\spc{4}

\intermediatesubproblem{Prove the PMF of $T = X_1 + X_2$ from (b) using a convolution. Show that the function reduces to the piecewise function in (b) when substituting for the values of $t$.}\spc{6}


\hardsubproblem{Let 

\beqn
X_1 \sim \begin{cases}
3 \withprob 0.3 \\
6 \withprob 0.7
\end{cases} \quad \text{independent of} \quad
%
X_2 \sim \begin{cases}
4 \withprob 0.4 \\
8 \withprob 0.6
\end{cases} 
\eeqn

Find the PMF of $T = X_1 + X_2$ using a convolution.}\spc{6}


\hardsubproblem{Prove the PMF of a binomial inductively using convolutions on the sequence of r.v.'s $\Xoneton \iid \bernoulli{p}$. You will need to use Pascal's Triangle combinatorial identity we employed in class.}\spc{7}


\extracreditsubproblem{Prove the PMF of a negative binomial inductively using convolutions on the sequence of r.v.'s $\Xoneton \iid \geometric{p}$. You will need to use the \href{https://en.wikipedia.org/wiki/Hockey-stick_identity}{hockey stick identity}.}\spc{7}

\hardsubproblem{Let $X_1 \sim \binomial{n_1}{p}$ independent of $X_2 \sim \binomial{n_2}{p}$. Find the PMF of the sum of $T = X_1 + X_2$ using a convolution.}\spc{13}


\easysubproblem{Prove the PMF of $X \sim \poisson{\lambda}$ using the limit as $n \rightarrow \infty$ and let $p = \overn{\lambda}$.}\spc{9}


\hardsubproblem{Let $X_1 \sim \poisson{\lambda_1}$ independent of $X_2 \sim \poisson{\lambda_2}$. Find the PMF of the sum of $T = X_1 + X_2$ using a convolution.}\spc{9}



\end{enumerate}


\problem{These exercises introduce probabilities of conditional subsets of the supports of multiple r.v.'s.}


\begin{enumerate}

\hardsubproblem{Let $X \sim \geometric{p_x}$ independent of $Y \sim \geometric{p_y}$. Find $\prob{X > Y}$.}\spc{10}


\hardsubproblem{Given (a), find $\prob{X = Y}$.}\spc{10}

\hardsubproblem{As both $p_x$ and $p_y$ are reduced to zero, but $r = \frac{p_x}{p_y}$, what is the asymptotic probability you found in (a)?}\spc{10}


\easysubproblem{Without needing to compute (c), what is the answer if $r=1$?}\spc{0}

\intermediatesubproblem{Let $X \sim \poisson{\lambda}$ independent of $Y \sim \poisson{\lambda}$. Find an expression for $\prob{X > Y}$ \emph{as best as you are able to answer}. Part of this exercise is identifying where you cannot go any further.}\spc{9}


\end{enumerate}


\problem{These exercises will introduce the Multinomial distribution.}


\begin{enumerate}

\easysubproblem{If $\X \sim \multinomial{n}{\p}$ where $\dime{\X} = k$, what is $\dime{\p}$?}\spc{0}


\easysubproblem{If $\X \sim \multinomial{n}{\p}$ and $n= 10$ and $\dime{\X} = 15$ as a column vector, give an example value of $\x$, a realization of the r.v. $\X$.}\spc{2}


\intermediatesubproblem{If $\X \sim \multinomial{n}{\p}$ and $n= 10$ and $\p = \bracks{0.2, 0.8}^\top$, find $\muvec := \expe{\X}$.}\spc{2}


\hardsubproblem{If $\X_1 \sim \multinomial{n}{\p}$ and independently $\X_2 \sim \multinomial{n}{\p}$ where $\dime{\X_1} = \dime{\X_2} = k$. Find the JMF of $\T_2 = \X_1 + \X_2$ from the definition of convolution. This looks harder than it is! First, use the definition of convolution. Then follow 1(g) updating each piece of information from binomial to multinomial, Finally, use Theorem 1 in \href{http://www.lrecits.usthb.dz/1.3.pdf}{this paper} for the summation.}\spc{9}

\end{enumerate}


\end{document}

