\documentclass[12pt]{article}

\include{preamble}

\title{Math 621 Fall 2017 \\ Midterm Examination One}
\author{Professor Adam Kapelner}

\date{October 3, 2017}

\begin{document}
\maketitle

\noindent Full Name \line(1,0){410}

\thispagestyle{empty}

\section*{Code of Academic Integrity}

\footnotesize
Since the college is an academic community, its fundamental purpose is the pursuit of knowledge. Essential to the success of this educational mission is a commitment to the principles of academic integrity. Every member of the college community is responsible for upholding the highest standards of honesty at all times. Students, as members of the community, are also responsible for adhering to the principles and spirit of the following Code of Academic Integrity.

Activities that have the effect or intention of interfering with education, pursuit of knowledge, or fair evaluation of a student's performance are prohibited. Examples of such activities include but are not limited to the following definitions:

\paragraph{Cheating} Using or attempting to use unauthorized assistance, material, or study aids in examinations or other academic work or preventing, or attempting to prevent, another from using authorized assistance, material, or study aids. Example: using an unauthorized cheat sheet in a quiz or exam, altering a graded exam and resubmitting it for a better grade, etc.
\\

\noindent I acknowledge and agree to uphold this Code of Academic Integrity. \\

\begin{center}
\line(1,0){250} ~~~ \line(1,0){100}\\
~~~~~~~~~~~~~~~~~~~~~signature~~~~~~~~~~~~~~~~~~~~~~~~~~~~~~~~~~~~~~~~~~~~~ date
\end{center}

\normalsize

\section*{Instructions}

This exam is seventy five minutes and closed-book. You are allowed one page (front and back) of a \qu{cheat sheet.} You may use a graphing calculator of your choice. Please read the questions carefully. If the question reads \qu{compute,} this means the solution will be a number otherwise you can leave the answer in choose, permutation, exponent, factorial or any other notation which could be resolved to a number with a computer. I advise you to skip problems marked \qu{[Extra Credit]} until you have finished the other questions on the exam, then loop back and plug in all the holes. I also advise you to use pencil. The exam is 100 points total plus extra credit. Partial credit will be granted for incomplete answers on most of the questions. \fbox{Box} in your final answers. Good luck!

\pagebreak

\problem Below are some theoretical exercises.


\benum
\subquestionwithpoints{7}  Assume independent r.v.'s $X$ and $Y$ where $\support{X} = (0, \infty)$ and $\support{Y} = (0, \infty)$. Beginning from the general definition of convolution, prove that

\beqn
f_{X + Y}(t) =  \int\displaylimits_0^t f_X(x) f_Y(t-x) dx
\eeqn

where by convention, the notation $f_X$ represents the PDF of r.v. $X$ which \emph{does not include} an indicator function.
 \spc{8}

\subquestionwithpoints{7}  If $X,Y \iid \exponential{\lambda}$, find $\prob{X \geq Y}$. \spc{10}

\subquestionwithpoints{10}  If $X,Y \iid \exponential{\lambda}$, find the conditional density of $X$ given $X+Y$. There is no guarantee that the result will be the density of a brand name r.v., but if it is, denote it and find the parameter(s) as a function of $\lambda$. \spc{10}


\subquestionwithpoints{7}  Let $X \sim \binomial{n_1}{p_1}$ independent of $Y \sim \binomial{n_2}{p_2}$. Find the PMF of $X+Y$ \textit{as best as you possibly can} (even if there is no closed form solution). \spc{10}

\subquestionwithpoints{7}  Let $X \sim \binomial{2000}{0.004}$ independent of $Y \sim \binomial{20000}{0.0004}$. Approximate $\prob{X + Y = 0}$. The correct answer uses one simple operation and no credit will be given for answers that require the use of non-trivial computing. \spc{10}

\subquestionwithpoints{5}  Let $X \sim \negbin{35}{0.37}$ independent of $Y \sim \geometric{0.37}$. Find the PMF of $X + Y$. \spc{10}

\subquestionwithpoints{5} Calculate $\gamma(5,24.5) + \displaystyle\int\displaylimits_{24.5}^\infty t^4 e^{-t}dt$ as a number $\in \naturals$ explicitly. \spc{10}

\eenum

\problem Below are some questions about waiting times.


\benum

\subquestionwithpoints{5} The time until phone the next phone call is exponentially distributed with an average of half hour. If you have already waited half hour, find the probability you will wait more than another half hour. \spc{10}

\subquestionwithpoints{5} The time until phone the next phone call is exponentially distributed with an average of half hour. What is the probability you get two phone calls in one hour? \spc{10}

\subquestionwithpoints{7} The time until phone the next phone call is exponentially distributed with an average of half hour. What is the probability you get 10 phone calls before 5hr? You can answer using the CDF of a r.v. you define.\spc{10}

\eenum

\problem Below are some theoretical exercises about the vector-valued r.v.'s.


\benum
\subquestionwithpoints{6} Let $\Xoneton \iid \binomial{n}{p}$ and let $\X$ denote the vector of these r.v.'s. Find $\var{\X}$. \spc{10}

\subquestionwithpoints{7} Let $\Xoneton \iid \binomial{n_0}{p}$ and let $\X$ denote the vector of these r.v.'s. Is $\X \sim \multinomial{m}{\p}$? If yes, find the values of its parameters, $m$ and $\p$ as functions of $n_0$ and $p$. If no, explain why not. \spc{10}

\subquestionwithpoints{7} A person goes to the grocery store and buys $n$ fruits. For each of his $n$ selections, he picks rambutans with probability $p$ otherwise he picks dragonfruits. If rambutans cost $a$ and dragonfruits cost $b$, find his expected bill. \spc{4}

\subquestionwithpoints{15} Find the standard deviation of his bill and simplify as best as possible.\spc{10}
\eenum

\end{document}