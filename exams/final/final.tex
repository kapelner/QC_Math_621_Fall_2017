\documentclass[12pt]{article}

\include{preamble}

\title{Math 621 Fall 2017 \\ Final Examination}
\author{Professor Adam Kapelner}

\date{December 19, 2017}

\begin{document}
\maketitle

\noindent Full Name \line(1,0){410}% ~~~ Section (A, B or C)~ \line(1,0){30}

\thispagestyle{empty}

\section*{Code of Academic Integrity}

\footnotesize
Since the college is an academic community, its fundamental purpose is the pursuit of knowledge. Essential to the success of this educational mission is a commitment to the principles of academic integrity. Every member of the college community is responsible for upholding the highest standards of honesty at all times. Students, as members of the community, are also responsible for adhering to the principles and spirit of the following Code of Academic Integrity.

Activities that have the effect or intention of interfering with education, pursuit of knowledge, or fair evaluation of a student's performance are prohibited. Examples of such activities include but are not limited to the following definitions:

\paragraph{Cheating} Using or attempting to use unauthorized assistance, material, or study aids in examinations or other academic work or preventing, or attempting to prevent, another from using authorized assistance, material, or study aids. Example: using an \emph{unauthorized} cheat sheet in a quiz or exam, altering a graded exam and resubmitting it for a better grade, etc.
\\

\noindent I acknowledge and agree to uphold this Code of Academic Integrity. \\

\begin{center}
\line(1,0){250} ~~~ \line(1,0){100}\\
~~~~~~~~~~~~~~~~~~~~~signature~~~~~~~~~~~~~~~~~~~~~~~~~~~~~~~~~~~~~~~~~~~~~ date
\end{center}

\normalsize

\section*{Instructions}

This exam is 120 minutes and closed-book. You are allowed three 8.5'' $\times$ 11'' page (front and back) of a \qu{cheat sheet.} You may use a graphing calculator of your choice. Please read the questions carefully. If the question reads \qu{compute,} this means the solution will be a number otherwise you can leave the answer in choose, permutation, exponent, factorial or any other notation which could be resolved to a number with a computer. I advise you to skip problems marked \qu{[Extra Credit]} until you have finished the other questions on the exam, then loop back and plug in all the holes. I also advise you to use pencil. The exam is 100 points total plus extra credit. Partial credit will be granted for incomplete answers on most of the questions. \fbox{Box} in your final answers. Good luck!

\pagebreak

\problem Below are some theoretical questions.

\benum


\subquestionwithpoints{6} Let $U = F_X(X)$ where $F_X$ denotes the CDF of $X$, a continuous r.v. Show that $U \sim \stduniform$.  Hint: do not use the usual transformation of variables formula but begin with $F_U(u) = \prob{U \leq u}$ instead. \spc{6} %mid2

\subquestionwithpoints{10} If $X, Y \iid \stdnormnot$, show that $\prob{X < Y} = \half$. Get as far as you can. \spc{10} %mid1

\subquestionwithpoints{4} If $\X \sim \multinomial{6}{\bracks{\third ~ \half ~ \sixth}^\top}$, what is $\prob{\X = \bracks{3 ~3 ~3}^\top}$? \spc{6} %mid1

\subquestionwithpoints{6} If $\X \sim \multinomial{6}{\bracks{\third ~ \half ~ \sixth}^\top}$, what is $\corr{X_1}{X_2}$? Simplify but do not compute explicitly. \spc{3} %mid1



\subquestionwithpoints{5} If $X_1, ~X_2 \iid \stduniform$ and $T = X_1 + X_2$, fill in the square below with one of the following symbols: \qu{$=,~\geq,~\leq,~>,~<$} or write \qu{?} if it cannot be determined given the information provided.

\beqn
\prob{T \in \bracks{\half,~\frac{3}{2}}} ~~ \text{\Huge{$\square$}} ~~ \prob{T \notin \bracks{\half,~\frac{3}{2}}}
\eeqn~\spc{-1} %mid1


\subquestionwithpoints{10} If $\Xoneton \iid \stduniform$, prove that $X_{max} \convd 1$. \spc{10}


\subquestionwithpoints{5} If $\Xoneton \iid \stduniform$ and $X_{(2)}$ is the \emph{second smallest} $X$, what is the explicit PDF of $X_{(2)}$? \spc{1} %mid2 


\subquestionwithpoints{10} [Extra credit] If $\Xoneton \iid \stduniform$ and $Y$ is the \emph{second smallest} $X$, show that $Y \convp 0$ without using the fact that $\forall r \geq 1~Y \convLp{r} 0 \Rightarrow Y \convp 0$. Leave this question for last. \spc{7}


\subquestionwithpoints{6} For discrete r.v. $X$, prove its ch.f. $\phi_X(t)$ exists for all $t$. \spc{5} 



\subquestionwithpoints{4} Let $\Zoneton \iid \stdnormnot$ and let $\Z = \bracks{Z_1 ~ Z_2 ~ \ldots Z_n}^\top$. If 

\beqn
X = \sqrt{2}\frac{Z_1}{\sqrt{Z_2^2 + Z_3^2}}\hspace{300px}
\eeqn 

find the distribution of $X$. You do not need to provide the PDF, just the name of the distribution and its parameter(s). Write it above. \spc{0} 

\subquestionwithpoints{4} Let $\Zoneton \iid \stdnormnot$ and let $\Z = \bracks{Z_1 ~ Z_2 ~ \ldots Z_n}^\top$. If 

\beqn
X = 2\frac{Z_1^2}{{Z_2^2 + Z_3^2}}\hspace{300px}
\eeqn

find the distribution of $X$. You do not need to provide the PDF, just the name of the distribution and its parameter(s). Write it above. \spc{0} 


\subquestionwithpoints{4} Under what circumstance(s) is the following true?

\beqn
\frac{\Xbar - \mu}{\frac{S}{\sqrt{n}}} \sim T_{n-1}
\eeqn 

Note: $\Xbar$ is the estimator for the mean $\mu$ we discussed in class and $S$ is the biased estimator for the standard error $\sigma$ we discussed in class. This should be one sentence.\spc{1} 



\subquestionwithpoints{6} Let $\Xoneton \iid \normnot{\mu}{\sigsq}$. Find the PDF of $Y = \sum_{i=1}^n \squared{X_i - \Xbar}$. \spc{3} 

\subquestionwithpoints{4} Let $\Zoneton \iid \stdnormnot$ and let $\Z = \bracks{Z_1 ~ Z_2 ~ \ldots Z_n}^\top$. If $\X = A\Z$ where $A \in \reals^{n \times n}$ and $\rank{A} = n$, find the distribution of $\X$. You do not need to provide the PDF, just the name of the distribution and its parameter(s). If you are using vectors, be explicit in the dimensions. \spc{2} 

\subquestionwithpoints{4} Let $\Zoneton \iid \stdnormnot$ and let $\Z = \bracks{Z_1 ~ Z_2 ~ \ldots Z_n}^\top$. If $\X = A^\top \Z - \c$ where $A \in \reals^{n \times n}$,  $\rank{A} = n$ and $\c \in \reals^n$, find the distribution  $\X$. You do not need to provide the PDF, just the name of the distribution and its parameter(s). If you are using vectors, be explicit in the dimensions. \spc{2} 



\subquestionwithpoints{4} Let $\Zoneton \iid \stdnormnot$ and let $\Z = \bracks{Z_1 ~ Z_2 ~ \ldots Z_n}^\top$. If $\X = A\Z$ where $A \in \reals^{n \times n}$ and $\rank{A} = n$, find the distribution of $Y = \X^\top (A^{-1})^\top A^{-1} \X$. You do not need to provide the PDF, just the name of the distribution and its parameter(s). If you are using vectors, be explicit in the dimensions. \spc{2} 


\subquestionwithpoints{8} Prove that for any r.v. $X$ and any constant $a$ that 

\beqn
\prob{X \geq a} \leq \min_{t>0} \braces{ e^{-at} M_X(t)}
\eeqn

where $M_X(t)$ is the moment generating function of $X$. \spc{7} 


\subquestionwithpoints{8} Let $X_n \sim \normnot{\oneover{n}}{\squared{\oneover{n}}}$. Prove that $X_n \convLp{2} 0$.  \spc{3} 


\eenum


\end{document}
