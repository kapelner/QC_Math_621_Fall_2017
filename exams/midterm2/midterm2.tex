\documentclass[12pt]{article}

\include{preamble}

\title{Math 621 Fall 2017 \\ Midterm Examination Two}
\author{Professor Adam Kapelner}

\date{November 14, 2017}

\begin{document}
\maketitle

\noindent Full Name \line(1,0){410}% ~~~ Section (A, B or C)~ \line(1,0){30}

\thispagestyle{empty}

\section*{Code of Academic Integrity}

\footnotesize
Since the college is an academic community, its fundamental purpose is the pursuit of knowledge. Essential to the success of this educational mission is a commitment to the principles of academic integrity. Every member of the college community is responsible for upholding the highest standards of honesty at all times. Students, as members of the community, are also responsible for adhering to the principles and spirit of the following Code of Academic Integrity.

Activities that have the effect or intention of interfering with education, pursuit of knowledge, or fair evaluation of a student's performance are prohibited. Examples of such activities include but are not limited to the following definitions:

\paragraph{Cheating} Using or attempting to use unauthorized assistance, material, or study aids in examinations or other academic work or preventing, or attempting to prevent, another from using authorized assistance, material, or study aids. Example: using an \emph{unauthorized} cheat sheet in a quiz or exam, altering a graded exam and resubmitting it for a better grade, etc.
\\

\noindent I acknowledge and agree to uphold this Code of Academic Integrity. \\

\begin{center}
\line(1,0){250} ~~~ \line(1,0){100}\\
~~~~~~~~~~~~~~~~~~~~~signature~~~~~~~~~~~~~~~~~~~~~~~~~~~~~~~~~~~~~~~~~~~~~ date
\end{center}

\normalsize

\section*{Instructions}

This exam is seventy five minutes and closed-book. You are allowed one 8.5'' $\times$ 11'' page (front and back) of a \qu{cheat sheet.} You may use a graphing calculator of your choice. Please read the questions carefully. If the question reads \qu{compute,} this means the solution will be a number otherwise you can leave the answer in choose, permutation, exponent, factorial or any other notation which could be resolved to a number with a computer. I advise you to skip problems marked \qu{[Extra Credit]} until you have finished the other questions on the exam, then loop back and plug in all the holes. I also advise you to use pencil. The exam is 100 points total plus extra credit. Partial credit will be granted for incomplete answers on most of the questions. \fbox{Box} in your final answers. Good luck!

\pagebreak

\problem Below are some theoretical questions.

\benum

%\subquestionwithpoints{6} Let $X \sim \exponential{\lambda}$ and $Y = \sqrt{X}$. Find $f_Y(y)$ and its support. If $Y$ is a brand name r.v., indicate it with the notation used in class. \spc{6}


\subquestionwithpoints{10} Let $X \sim \betanot{\alpha}{\beta}$ and $Y = 1 - X$. (i) Find $\support{Y}$. (ii) Find $f_Y(y)$. (iii) Simplify $f_Y(y)$. (iv) If $Y$ is a brand name r.v., indicate it with the notation used in class. \spc{10}

\subquestionwithpoints{10} Let $X \sim \logistic{0}{1}$ and $Y = e^X$ i.e. the log-logistic distribution analogous to the log-normal distribution. (i) Find $\support{Y}$. (ii) Find $f_Y(y)$. (iii) Simplify $f_Y(y)$. \spc{10}

\subquestionwithpoints{6} If $Y~|~X = x \sim \normnot{0}{\oneover{x}}$ and $X \sim \exponential{1}$, draw a tree diagram for $X,Y$.\spc{4}

\subquestionwithpoints{10} As given previously, if $Y~|~X = x \sim \normnot{0}{\oneover{x}}$ and $X \sim \exponential{1}$. (i) Find $\support{Y}$. (ii) Find $f_Y(y)$. (iii) Simplify $f_Y(y)$. (iv) If $Y$ is a brand name r.v., indicate it with the notation used in class. \spc{16}


\subquestionwithpoints{10} Let $X \sim \laplace{0}{1}$ and $Y = X\indic{X \geq 0}$. (i) Find $\support{Y}$. (ii) Find $f_Y(y)$. (iii) Simplify $f_Y(y)$. (iv) If $Y$ is a brand name r.v., indicate it with the notation used in class. \spc{10}


\subquestionwithpoints{10} Consider modeling human survival (with unit years) by $T \sim \weibull{\lambda}{k}$. Make up values of $k$ and $\lambda$ that are appropriate for this modeling challenge. Note that $\expe{T} = \oneover{\lambda}\Gammaf{1 + \oneover{k}}$. Explain why you chose these $k$ and $\lambda$ values. \spc{10}


\subquestionwithpoints{4} Why wouldn't $W \sim \gumbel{\mu}{\beta}$ be appropriate for modeling human survival (with unit years)? \spc{2}


\subquestionwithpoints{12} If $Z_1, Z_2 \iid \stdnormnot$ and $Y = \squared{Z_1 / Z_2}$. (i) Find $\support{Y}$. (ii) Find $f_Y(y)$. (iii) Simplify $f_Y(y)$. (iv) If $Y$ is a brand name r.v., indicate it with the notation used in class. \spc{5}


\subquestionwithpoints{12} Let $\Xoneton \iid \gammadist{1}{\beta}$ and $Y = \min{\Xoneton}$. (i) Find $\support{Y}$. (ii) Find $f_Y(y)$. (iii) Simplify $f_Y(y)$. (iv) If $Y$ is a brand name r.v., indicate it with the notation used in class. \spc{12}

\subquestionwithpoints{12} Let $X \sim \stduniform$ independent of $Y \sim \stduniform$ and $R = \frac{X}{Y}$. (i) Find $\support{R}$. (ii) Find $f_R(r)$. (iii) Simplify $f_R(r)$. (iv) If $R$ is a brand name r.v., indicate it with the notation used in class. \spc{11}




%\subquestionwithpoints{4} Evaluate $I_{\half}(2,2)$. \spc{3}


\eenum

\problem There are 5 class lectures left in Math 621. Write below about what kind of probability material \emph{you} want to learn. \ingray{[4 pt / 100 pts]}

\end{document}
